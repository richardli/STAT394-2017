\documentclass[usenames,dvipsnames,12pt,compress, final, handout]{beamer}\usepackage[]{graphicx}\usepackage[]{color}
\usepackage{alltt} 
\let\Tiny=\tiny
\usepackage{eqnarray,amsmath}
% \usepackage{wasysym}
\usepackage{mathtools}
\usepackage{multirow}
\usepackage{booktabs}
\usepackage{tabularx}
\usepackage{array}
\usepackage{multirow}
% \usepackage{bigstrut}
\usepackage{graphicx}
\usepackage[round]{natbib}
\usepackage{bm}
\usepackage{tikzsymbols}
\setbeamersize{text margin left=15pt,text margin right=10pt}
\usepackage{tikzsymbols}
\usepackage{textcomp}
\usepackage{parskip}
\setbeamertemplate{navigation symbols}{}    
\setbeamertemplate{footline}[frame number]{}
\usepackage{soul}
\usetheme{Singapore}
\usecolortheme{lily}
%Misc Commands
\newcommand{\mbf}{\mathbf}
\newcommand{\lexp}{$\overset{\mbox{\tiny 0}}{e}$}
\newenvironment{wideitemize}{\itemize\addtolength{\itemsep}{5pt}}{\enditemize}


\newcommand{\bx}{{\bm x}}
\newcommand{\bX}{{\bm X}}
\newcommand{\by}{{\bm y}}
\newcommand{\bY}{{\bm Y}}
\newcommand{\bW}{{\bm W}}
\newcommand{\bG}{{\bm G}}
\newcommand{\bR}{{\bm R}}
\newcommand{\bZ}{{\bm Z}}
\newcommand{\bV}{{\bm V}}
\newcommand{\bL}{{\bm L}}
\newcommand{\bz}{{\bm z}}
\newcommand{\be}{{\bm e}}
\newcommand{\bgamma}{{\bm \gamma}}
\newcommand{\bbeta}{{\bm \beta}}
\newcommand{\balpha}{{\bm \alpha}}
\newcommand{\bSigma}{{\bm \Sigma}}
\newcommand{\bmu}{{\bm \mu}}
\newcommand{\btheta}{{\bm \theta}}
\newcommand{\bepsilon}{{\bm \epsilon}}
\newcommand{\bone}{{\bm 1}}
\newcommand{\bzero}{{\bm 0}}
\newcommand{\bC}{{\bm C}}
\newcommand{\bI}{{\bm I}}
\newcommand{\bA}{{\bm A}}
\newcommand{\bB}{{\bm B}}
\newcommand{\bQ}{{\bm Q}}
\newcommand{\bS}{{\bm S}}
\newcommand{\bD}{{\bm D}}
\newcommand{\cQ}{\mathcal{Q}}
\newcommand{\cR}{\mathcal{R}}
\newcommand{\cU}{\mathcal{U}}
\newcommand{\cI}{\mathcal{I}}
\newcommand{\cL}{\mathcal{L}}
\newcommand{\RR}{\mathbb{R}}
\newcommand{\orange}{\textcolor{Orange}}
\newcommand{\green}{\textcolor{green}}
\newcommand{\blue}{\textcolor{blue}}
\newcommand{\red}{\textcolor{red}}
\newcommand{\purple}{\textcolor{purple}}
\newcommand{\gray}{\textcolor{gray}}
\newcommand{\ok}{\nonumber}

% Adjust vertical spacing in lists
\makeatletter
\def\@listi{\leftmargin\leftmargini
            \topsep     8\p@ \@plus2\p@ \@minus2.5\p@
            \parsep     0\p@
            \itemsep  5\p@ \@plus2\p@ \@minus3\p@}
\let\@listI\@listi
\def\@listii{\leftmargin\leftmarginii
              \topsep    6\p@ \@plus1\p@ \@minus2\p@
              \parsep    0\p@ \@plus\p@
              \itemsep  3\p@ \@plus2\p@ \@minus3\p@}
\def\@listiii{\leftmargin\leftmarginiii
              \topsep    3\p@ \@plus1\p@ \@minus2\p@
              \parsep    0\p@ \@plus\p@
              \itemsep  2\p@ \@plus2\p@ \@minus3\p@}
\makeatother
% Dealing with fraile envrionment of beamer with codes
\newenvironment{xframe}[2][]
  {\begin{frame}[fragile,environment=xframe,#1]
  \frametitle{#2}}
  {\end{frame}}



\title{Stat 394 Probability I}
\subtitle{Lecture 4}

\author[]{Richard Li}
\date{\today}
\IfFileExists{upquote.sty}{\usepackage{upquote}}{}
\begin{document}
% \renewcommand{\itemize}[1][<+(1)->]{\olditemize[#1]}

\maketitle

%================================================================%
\section{Independence}
\stepcounter{subsection}
\frame{
  \frametitle{Definition}
}
\frame{
  \frametitle{Example}
  Roll a die twice, consider the event $E_1 = \{\mbox{sum of the two values is 6}\}$, and $E_2 = \{\mbox{first value is 4}\}$.
  \vspace{6cm}
}

\frame{
  \frametitle{Proposition}
  If event $E$ and $F$ are independent, then $E$ and $F^c$ are also independent
    \vspace{6cm}

}

\frame{
  \frametitle{Example}
  You roll a die, your friend tosses a coin. (1) If you roll 6, you win outright. (2) If you do not roll 6 and your friend tosses Heads, you lose outright. (3) If neither, the game is repeated until decided.
  \vspace{5cm}
}

\frame{
  \frametitle{Pepys-Newton Problem (1693)}
  Which of the following is most likely?
\begin{itemize}
  \item 6 fair dice are tossed independently and at
least one 6 appears.
  \item 12 fair dice are tossed independently and
at least two 6's appear.
  \item 18 fair dice are tossed independently and
at least three 6's appear.
\end{itemize}
\vspace{3cm}
}

\frame{
  \frametitle{Multiple events}
}

\frame{
  \frametitle{Mutual Independence}
}
\frame{
  \frametitle{Proposition}
  If E, F, and G are mutually independent, then E is also independent of $F \cup G$.
  \vspace{6cm}
}

\frame{
\frametitle{Example}

}


%================================================================%
\section{Conditional probabilities}
\stepcounter{subsection}
\frame{
  \frametitle{Proposition}
  Once we start to think about conditional relationships involving more than two events, there is a helpful fact to know about: Conditional probabilities are probabilities. 
  \vspace{6cm}
}
\frame{
  \frametitle{Proposition}
}
\frame{
  \frametitle{Monty Hall problem}
}
\frame{
  \frametitle{Prosecutor's fallacy}
}
\frame{
  \frametitle{Other common mistakes}
}
\frame{
  \frametitle{More on conditional independence}
}
\frame{
  \frametitle{Simpson's paradox}
}
\frame{
  \frametitle{Simpson's paradox}
}

\end{document}