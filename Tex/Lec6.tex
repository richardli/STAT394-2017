\documentclass[usenames,dvipsnames,12pt,compress, final, handout]{beamer}\usepackage[]{graphicx}\usepackage[]{color}
\usepackage{alltt} 
\let\Tiny=\tiny
\usepackage{eqnarray,amsmath}
% \usepackage{wasysym}
\usepackage{mathtools}
\usepackage{multirow}
\usepackage{booktabs}
\usepackage{tabularx}
\usepackage{array}
\usepackage{multirow}
% \usepackage{bigstrut}
\usepackage{graphicx}
\usepackage[round]{natbib}
\usepackage{bm}
\usepackage{tikzsymbols}
\setbeamersize{text margin left=15pt,text margin right=10pt}
\usepackage{tikzsymbols}
\usepackage{textcomp}
\usepackage{parskip}
\setbeamertemplate{navigation symbols}{}    
\setbeamertemplate{footline}[frame number]{}
\usepackage{soul}
\usetheme{Singapore}
\usecolortheme{lily}
%Misc Commands
\newcommand{\mbf}{\mathbf}
\newcommand{\lexp}{$\overset{\mbox{\tiny 0}}{e}$}
\newenvironment{wideitemize}{\itemize\addtolength{\itemsep}{5pt}}{\enditemize}


\newcommand{\bx}{{\bm x}}
\newcommand{\bX}{{\bm X}}
\newcommand{\by}{{\bm y}}
\newcommand{\bY}{{\bm Y}}
\newcommand{\bW}{{\bm W}}
\newcommand{\bG}{{\bm G}}
\newcommand{\bR}{{\bm R}}
\newcommand{\bZ}{{\bm Z}}
\newcommand{\bV}{{\bm V}}
\newcommand{\bL}{{\bm L}}
\newcommand{\bz}{{\bm z}}
\newcommand{\be}{{\bm e}}
\newcommand{\bgamma}{{\bm \gamma}}
\newcommand{\bbeta}{{\bm \beta}}
\newcommand{\balpha}{{\bm \alpha}}
\newcommand{\bSigma}{{\bm \Sigma}}
\newcommand{\bmu}{{\bm \mu}}
\newcommand{\btheta}{{\bm \theta}}
\newcommand{\bepsilon}{{\bm \epsilon}}
\newcommand{\bone}{{\bm 1}}
\newcommand{\bzero}{{\bm 0}}
\newcommand{\bC}{{\bm C}}
\newcommand{\bI}{{\bm I}}
\newcommand{\bA}{{\bm A}}
\newcommand{\bB}{{\bm B}}
\newcommand{\bQ}{{\bm Q}}
\newcommand{\bS}{{\bm S}}
\newcommand{\bD}{{\bm D}}
\newcommand{\cQ}{\mathcal{Q}}
\newcommand{\cR}{\mathcal{R}}
\newcommand{\cU}{\mathcal{U}}
\newcommand{\cI}{\mathcal{I}}
\newcommand{\cL}{\mathcal{L}}
\newcommand{\RR}{\mathbb{R}}
\newcommand{\orange}{\textcolor{Orange}}
\newcommand{\green}{\textcolor{green}}
\newcommand{\blue}{\textcolor{blue}}
\newcommand{\red}{\textcolor{red}}
\newcommand{\purple}{\textcolor{purple}}
\newcommand{\gray}{\textcolor{gray}}
\newcommand{\ok}{\nonumber}

% Adjust vertical spacing in lists
\makeatletter
\def\@listi{\leftmargin\leftmargini
            \topsep     8\p@ \@plus2\p@ \@minus2.5\p@
            \parsep     0\p@
            \itemsep  5\p@ \@plus2\p@ \@minus3\p@}
\let\@listI\@listi
\def\@listii{\leftmargin\leftmarginii
              \topsep    6\p@ \@plus1\p@ \@minus2\p@
              \parsep    0\p@ \@plus\p@
              \itemsep  3\p@ \@plus2\p@ \@minus3\p@}
\def\@listiii{\leftmargin\leftmarginiii
              \topsep    3\p@ \@plus1\p@ \@minus2\p@
              \parsep    0\p@ \@plus\p@
              \itemsep  2\p@ \@plus2\p@ \@minus3\p@}
\makeatother
% Dealing with fraile envrionment of beamer with codes
\newenvironment{xframe}[2][]
  {\begin{frame}[fragile,environment=xframe,#1]
  \frametitle{#2}}
  {\end{frame}}



\title{Stat 394 Probability I}
\subtitle{Lecture 6}

\author[]{Richard Li}
\date{\today}
\IfFileExists{upquote.sty}{\usepackage{upquote}}{}
\begin{document}
% \renewcommand{\itemize}[1][<+(1)->]{\olditemize[#1]}

\maketitle

%================================================================%
\section{Poisson random variable}
\stepcounter{subsection}
 \frame{
 \frametitle{Review: binomial distribution}
 }

 \frame{
 \frametitle{Theorem}
 If $X \sim Bin(n, p)$, then as $n \rightarrow \infty$, 
 \[
  P(X = i) \rightarrow \frac{e^{-\lambda}\lambda^i}{i!}, \;\; \mbox{for } i = 0, 1, 2, ...
 \]
 where $\lambda = np$.
 \vspace{4cm}
 }
 \frame{
 \frametitle{Theorem}
 }
 \frame{
 \frametitle{Poisson distribution}
 Definition: We say $X \sim Poisson(\lambda)$ with $\lambda > 0$ if it has the following p.m.f 
 \[
  P(X = i) = \frac{e^{-\lambda}\lambda^i}{i!}, \;\; \mbox{for } i = 0, 1, 2, ...
 \]  

\vspace{1cm}

Example: Suppose the number of typos on a single page of my lecture slides has a Poisson distribution with parameter $\lambda = 0.2$. Calculate the probability that there is at leest one typo on this page.
\vspace{3cm}
}
\frame{
  \frametitle{Example}
  Suppose that the probability that a person is killed by lightning in a year is, independently, 1/(500 million). Assume that the US population is 300 million. 
  \begin{itemize}
    \item Compute P(3 or more people will be killed by lightning next year) exactly.
    \vspace{2cm}
    \item Approximate the above probability.
    \vspace{2cm}
  \end{itemize}
}
\frame{
  \frametitle{Expectation}
}
\frame{
  \frametitle{Variance}
}
\frame{
  \frametitle{Variance}
}

%================================================================%
\section{Poisson paradigm}
\stepcounter{subsection}
\frame{
  \frametitle{Poisson paradigm}
  Consider $n$ events, with $p_i$ equal to the probability that event $i$ occurs, $i = 1, ..., n$. If all the $p_i$ are ``small'' and the trials are either independent or at most ``weakly dependent'', then the number of these events that occur approximately has a Poisson distribution with mean $\lambda = \sum_{i=1}^n p_i$.
}

\frame{
  \frametitle{Examples}
\begin{itemize}
  \item the number of winners in a lottery
  \item the number of people entering a store between 9:00 and 10:00 on weekdays
  \item the number of wrong numbers dialed each day in Seattle
  \item the number of raindrops on one brick on the red square at a given second
\end{itemize}

}

\frame{
  \frametitle{Example}
  A company which sells flood insurance has three groups of clients. 
  \begin{itemize}
    \item The first group, $N_1 = 10000$, is low-risk: each client has probability $p_1 = 0.01\%$ of a flood, independently of others.
    \item The second group, $N_2 = 1000$ clients, is medium-risk: $p_2 = 0.05\%$.
    \item The third group, $N_3 = 100$ clients, is high-risk, $p_3 = 0.5\%$.
  \end{itemize}
  For every flood, a company pays $100,000$. How much should it charge its clients so that is does not go bankrupt with probability at least $95\%$?The third group, $N_3 = 100$ clients, is high-risk, $p_3 = 0.5\%$
\vspace{3cm}
}

\frame{
  \frametitle{Example: People v. Collins, 1968}
  Suppose there are $n = 5$ million couples in the LA area, and the probability that a randomly chosen couple fits the descriptions of the witness is $p = 1/12 \mbox{ million}$. One such couple is found. What is the probability that there exist other couple(s) who also fit the descriptions.  
  \vspace{2cm}
}

\frame{
\frametitle{Continuous time interpretation of Poisson r.v.}
  For events happening in continuous time at a rate of $\lambda$, the number of event happening at any interval of length $t$ follows a Poisson distribution with parameter $\lambda t$ if the following (informal) conditions are satisfied:
  \begin{enumerate}
    \item The probability of exactly one occurrence of the event in a given time
interval $h$ is $\lambda h + o(h)$.
    \item  The probability of two or more occurrences of the event in a very small
time interval is negligible.
    \item The numbers of occurrences of the event in disjoint time intervals are
mutually independent.

  \end{enumerate}
}
\frame{
  \frametitle{Example}
  \begin{itemize}
    \item Number of flying-bomb hits in the south of London during World
War II
    \item Number of wars per year
  \item Number of earthquakes occurring during a fixed time span
  \end{itemize}
}
\frame{
  \frametitle{Example}
   Suppose that the probability that a person is killed by lightning in a year is, independently, 1/(500 million). Assume that the US population is 300 million. 
  \begin{itemize}
    \item Approximate P(two or more people are killed by lightning within the first 6 months of next year).
    \vspace{2cm}
  \end{itemize}
}
\frame{
  \frametitle{Combining multiple distributions together}
  \begin{itemize}
    \item Approximate P(in exactly 3 of next 10 years exactly 3 people are killed by lightning).
    \vspace{2cm}
    \item Compute the expected number of years, among the next 10, in which 2 or more people are killed by lightning.
        \vspace{2cm}
  \end{itemize}
}

\frame{
  \frametitle{More example}
  The number of students coming to the Suzzallo and Allen library in a given hour follow a Poisson distribution with parameter $\lambda$. Each student choose to stay at Suzzallo library with probability $1/4$, or stay at Allen library with probability $3/4$. What is the distribution of the number of students in Allen library in a given hour?
  \vspace{3cm}
} 

\frame{
  \frametitle{More example}
 }

\frame{
  \frametitle{Final thoughts about Poisson distribution}
  \begin{itemize}
    \item This use of Poisson random variable in continuous time is often times referred to as ``Poisson process''.
    \item It goes much deeper beyond what we have looked at in this course.
          \begin{itemize}
            \item e.g., how do we characterize the interarrival times between two events? \textit{(You will learn in 395, I think)}  
          \end{itemize}
    \item It has a lot of cool applications in finance, physics, environmental science, etc.   
    \item The example we just talked about is called ``thinning/splitting of Poisson process''.   
  \end{itemize}
}

\end{document}