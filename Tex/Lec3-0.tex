\documentclass[usenames,dvipsnames,12pt,compress, final, handout]{beamer}\usepackage[]{graphicx}\usepackage[]{color}
\usepackage{alltt} 
\let\Tiny=\tiny
\usepackage{eqnarray,amsmath}
% \usepackage{wasysym}
\usepackage{mathtools}
\usepackage{multirow}
\usepackage{booktabs}
\usepackage{tabularx}
\usepackage{array}
\usepackage{multirow}
% \usepackage{bigstrut}
\usepackage{graphicx}
\usepackage[round]{natbib}
\usepackage{bm}
\usepackage{tikzsymbols}
\setbeamersize{text margin left=15pt,text margin right=10pt}
\usepackage{tikzsymbols}
\usepackage{textcomp}
\usepackage{parskip}
\setbeamertemplate{navigation symbols}{}    
\setbeamertemplate{footline}[frame number]{}
\usepackage{soul}
\usetheme{Singapore}
\usecolortheme{lily}
%Misc Commands
\newcommand{\mbf}{\mathbf}
\newcommand{\lexp}{$\overset{\mbox{\tiny 0}}{e}$}
\newenvironment{wideitemize}{\itemize\addtolength{\itemsep}{5pt}}{\enditemize}


\newcommand{\bx}{{\bm x}}
\newcommand{\bX}{{\bm X}}
\newcommand{\by}{{\bm y}}
\newcommand{\bY}{{\bm Y}}
\newcommand{\bW}{{\bm W}}
\newcommand{\bG}{{\bm G}}
\newcommand{\bR}{{\bm R}}
\newcommand{\bZ}{{\bm Z}}
\newcommand{\bV}{{\bm V}}
\newcommand{\bL}{{\bm L}}
\newcommand{\bz}{{\bm z}}
\newcommand{\be}{{\bm e}}
\newcommand{\bgamma}{{\bm \gamma}}
\newcommand{\bbeta}{{\bm \beta}}
\newcommand{\balpha}{{\bm \alpha}}
\newcommand{\bSigma}{{\bm \Sigma}}
\newcommand{\bmu}{{\bm \mu}}
\newcommand{\btheta}{{\bm \theta}}
\newcommand{\bepsilon}{{\bm \epsilon}}
\newcommand{\bone}{{\bm 1}}
\newcommand{\bzero}{{\bm 0}}
\newcommand{\bC}{{\bm C}}
\newcommand{\bI}{{\bm I}}
\newcommand{\bA}{{\bm A}}
\newcommand{\bB}{{\bm B}}
\newcommand{\bQ}{{\bm Q}}
\newcommand{\bS}{{\bm S}}
\newcommand{\bD}{{\bm D}}
\newcommand{\cQ}{\mathcal{Q}}
\newcommand{\cR}{\mathcal{R}}
\newcommand{\cU}{\mathcal{U}}
\newcommand{\cI}{\mathcal{I}}
\newcommand{\cL}{\mathcal{L}}
\newcommand{\RR}{\mathbb{R}}
\newcommand{\orange}{\textcolor{Orange}}
\newcommand{\green}{\textcolor{green}}
\newcommand{\blue}{\textcolor{blue}}
\newcommand{\red}{\textcolor{red}}
\newcommand{\purple}{\textcolor{purple}}
\newcommand{\gray}{\textcolor{gray}}
\newcommand{\ok}{\nonumber}

% Adjust vertical spacing in lists
\makeatletter
\def\@listi{\leftmargin\leftmargini
            \topsep 		8\p@ \@plus2\p@ \@minus2.5\p@
            \parsep 		0\p@
            \itemsep	5\p@ \@plus2\p@ \@minus3\p@}
\let\@listI\@listi
\def\@listii{\leftmargin\leftmarginii
              \topsep    6\p@ \@plus1\p@ \@minus2\p@
              \parsep    0\p@ \@plus\p@
              \itemsep  3\p@ \@plus2\p@ \@minus3\p@}
\def\@listiii{\leftmargin\leftmarginiii
              \topsep    3\p@ \@plus1\p@ \@minus2\p@
              \parsep    0\p@ \@plus\p@
              \itemsep  2\p@ \@plus2\p@ \@minus3\p@}
\makeatother
% Dealing with fraile envrionment of beamer with codes
\newenvironment{xframe}[2][]
  {\begin{frame}[fragile,environment=xframe,#1]
  \frametitle{#2}}
  {\end{frame}}



\title{Stat 394 Probability I}
\subtitle{Lecture 3}

\author[]{Richard Li}
\date{\today}
\IfFileExists{upquote.sty}{\usepackage{upquote}}{}
\begin{document}
% \renewcommand{\itemize}[1][<+(1)->]{\olditemize[#1]}

\maketitle

%================================================================%
% \section{Equally likely outcomes}
% \stepcounter{subsection}
% \frame{
% 	\frametitle{Review}
% 	\begin{itemize}
% 		\item Probability space\pause
% 		\item Set theory\pause
% 		\item Calculate probability when outcomes are equally likely
% 				\begin{itemize}
% 					\item find equally likely outcomes\pause
% 					\item use set theory to represent event $E$ using those outcomes\pause
% 					\item calculate probability of $E$
% 				\end{itemize} \pause	
% 		\item \textit{In many examples we have considered, the equally likely outcomes are obvious.}
% 	\end{itemize}
% }
% \frame{
% 	\frametitle{More examples}
% 	If 3 balls are ``randomly drawn'' from a bowl containing 6 white and 5 black balls, what is the probability that one of the balls is white and the other two black?
% 	\vspace{4cm}
% }
% \frame{
% 	\frametitle{More poker examples}
% 	A poker hand consists of 5 cards. If the cards have distinct consecutive values and are not all of the same suit, we say that the hand is a straight. What is the probability that one is dealt a straight? (suits: heart $\heartsuit$, diamond $\diamondsuit$, spade $\spadesuit$, club $\clubsuit$)
% 	\vspace{4cm}
% }
% \frame{
% 	\frametitle{More poker examples}
% 	What is the probability of a full house (three cards of the same value and two cards of another value)?
% 	\vspace{5cm}
% }
% \frame{
% 	\frametitle{More poker examples}
% 	A deck of 52 playing cards is shuffled, and the cards are turned up one at a time until the first ace appears. Is the next card, i.e., the card following the first ace, more likely to be the ace of hearts or the two of spades?
% 	\vspace{4cm}
% }


% \frame{
% 	\frametitle{Example: Birthday Problem}
% 	 Assume that there are k people in the room. What is the probability that there are two who share a birthday? We will ignore leap years, assume all birthdays are equally likely.
% 	\vspace{5cm}
% }

% \frame{
% 	\frametitle{More on the birthday problem}
% 	\begin{itemize}
% 		\item Each day, the Massachusetts lottery chooses a four digit number at random, with leading 0's allowed. 
% 		\item On February 6, 1978, the Boston Evening Globe reported that
% 	\end{itemize}
% 	\begin{quote}
% 			``During [the lottery’s] 22 months’ existence [...], no winning number has ever been repeated. [...] doesn’t expect to see duplicate winners until about half of the 10, 000 possibilities have been exhausted.''
% 	\end{quote}
% 	\begin{itemize}
% 		\item What if $k$ people enter the room one by one, which person has the highest probability of being the first to share the same birthday with the people in the room? 
% 	\end{itemize}
% }

% \frame{
% 	\frametitle{Coupon collector problem}
% 	Within the context of the birthday problem, assume that $k \geq n$ and compute $P(\mbox{all n birthdays are represented})$.
% 	\vspace{5cm}
% }

% \frame{
% 	\frametitle{Coupon collector problem}
	
% }

% \frame{
% 	\frametitle{Example: Was Klay Thompson in the zone?}
% 	On Dec 5, 2016, Klay Thompson scored 60 points, by making 31 of the 44 shots he took. These shots can be represented as 
% 	\[
% 		11011110010111111001110111101110111101010101
% 	\] 
% 	Let us simplify the problem: whenever a shot is made, we say the player has a ``hot hand'' and it stops when he starts to miss. For example, we say Klay had $12$ periods of hot hands in this game. Assuming any ordering of the $31$ made shots are equally likely in the $44$ shots, What's the probability of seeing $k$ periods of ``hot hand''?
% 	\vspace{3cm}
% }

% \frame{
% 	\frametitle{Example: Was Klay Thompson in the zone?}
% }
% \frame{
% 	\frametitle{Example: Was Klay Thompson in the zone?}
% }
% \frame{
% 	\frametitle{Example: Was Klay Thompson in the zone?}
% 	\includegraphics[width=\textwidth]{fun/KT.pdf}
% }
% \frame{
% 	\frametitle{Example: Was Klay Thompson in the zone?}
% 	\begin{itemize}
% 		\item Of course, this is an super over-simplified example
% 		\item A few limitations
% 			\begin{itemize}
% 				\item If you look at the first 17 shots, he seems more ``in the zone''.
% 				\item One score = hot hand? Not very appealing definition.
% 			\end{itemize}
% 		\item If you are interested: \url{https://arxiv.org/pdf/1706.03442.pdf} 
% 	\end{itemize}
% }

%================================================================%
\section{Probability through set functions}
\stepcounter{subsection}
\frame{
	\frametitle{Some definitions}
}
\frame{
	\frametitle{Proposition}
}
\frame{
	\frametitle{Proposition}
}
\frame{
	\frametitle{Proposition}
}
\frame{
	\frametitle{Proposition}
}
\frame{
	\frametitle{Example}
}

\end{document}