\documentclass[usenames,dvipsnames,12pt,compress, final, handout]{beamer}\usepackage[]{graphicx}\usepackage[]{color}
\usepackage{alltt} 
\let\Tiny=\tiny
\usepackage{eqnarray,amsmath}
% \usepackage{wasysym}
\usepackage{mathtools}
\usepackage{multirow}
\usepackage{booktabs}
\usepackage{tabularx}
\usepackage{array}
\usepackage{multirow}
% \usepackage{bigstrut}
\usepackage{graphicx}
\usepackage[round]{natbib}
\usepackage{bm}
\usepackage{tikzsymbols}
\setbeamersize{text margin left=15pt,text margin right=10pt}
\usepackage{tikzsymbols}
\usepackage{textcomp}
\usepackage{parskip}
\setbeamertemplate{navigation symbols}{}    
\setbeamertemplate{footline}[frame number]{}
\usepackage{soul}
\usetheme{Singapore}
\usecolortheme{lily}
%Misc Commands
\newcommand{\mbf}{\mathbf}
\newcommand{\lexp}{$\overset{\mbox{\tiny 0}}{e}$}
\newenvironment{wideitemize}{\itemize\addtolength{\itemsep}{5pt}}{\enditemize}


\newcommand{\bx}{{\bm x}}
\newcommand{\bX}{{\bm X}}
\newcommand{\by}{{\bm y}}
\newcommand{\bY}{{\bm Y}}
\newcommand{\bW}{{\bm W}}
\newcommand{\bG}{{\bm G}}
\newcommand{\bR}{{\bm R}}
\newcommand{\bZ}{{\bm Z}}
\newcommand{\bV}{{\bm V}}
\newcommand{\bL}{{\bm L}}
\newcommand{\bz}{{\bm z}}
\newcommand{\be}{{\bm e}}
\newcommand{\bgamma}{{\bm \gamma}}
\newcommand{\bbeta}{{\bm \beta}}
\newcommand{\balpha}{{\bm \alpha}}
\newcommand{\bSigma}{{\bm \Sigma}}
\newcommand{\bmu}{{\bm \mu}}
\newcommand{\btheta}{{\bm \theta}}
\newcommand{\bepsilon}{{\bm \epsilon}}
\newcommand{\bone}{{\bm 1}}
\newcommand{\bzero}{{\bm 0}}
\newcommand{\bC}{{\bm C}}
\newcommand{\bI}{{\bm I}}
\newcommand{\bA}{{\bm A}}
\newcommand{\bB}{{\bm B}}
\newcommand{\bQ}{{\bm Q}}
\newcommand{\bS}{{\bm S}}
\newcommand{\bD}{{\bm D}}
\newcommand{\cQ}{\mathcal{Q}}
\newcommand{\cR}{\mathcal{R}}
\newcommand{\cU}{\mathcal{U}}
\newcommand{\cI}{\mathcal{I}}
\newcommand{\cL}{\mathcal{L}}
\newcommand{\RR}{\mathbb{R}}
\newcommand{\orange}{\textcolor{Orange}}
\newcommand{\green}{\textcolor{green}}
\newcommand{\blue}{\textcolor{blue}}
\newcommand{\red}{\textcolor{red}}
\newcommand{\purple}{\textcolor{purple}}
\newcommand{\gray}{\textcolor{gray}}
\newcommand{\ok}{\nonumber}

% Adjust vertical spacing in lists
\makeatletter
\def\@listi{\leftmargin\leftmargini
            \topsep 		8\p@ \@plus2\p@ \@minus2.5\p@
            \parsep 		0\p@
            \itemsep	5\p@ \@plus2\p@ \@minus3\p@}
\let\@listI\@listi
\def\@listii{\leftmargin\leftmarginii
              \topsep    6\p@ \@plus1\p@ \@minus2\p@
              \parsep    0\p@ \@plus\p@
              \itemsep  3\p@ \@plus2\p@ \@minus3\p@}
\def\@listiii{\leftmargin\leftmarginiii
              \topsep    3\p@ \@plus1\p@ \@minus2\p@
              \parsep    0\p@ \@plus\p@
              \itemsep  2\p@ \@plus2\p@ \@minus3\p@}
\makeatother
% Dealing with fraile envrionment of beamer with codes
\newenvironment{xframe}[2][]
  {\begin{frame}[fragile,environment=xframe,#1]
  \frametitle{#2}}
  {\end{frame}}



\title{Stat 394 Probability I}
\subtitle{Lecture 3}

\author[]{Richard Li}
\date{\today}
\IfFileExists{upquote.sty}{\usepackage{upquote}}{}
\begin{document}
% \renewcommand{\itemize}[1][<+(1)->]{\olditemize[#1]}

\maketitle

%================================================================%
\section{Review and Roadmap}
\stepcounter{subsection}
\frame{
  \frametitle{What we have learned and what we are going to}
  \begin{itemize}
    \item Counting
        \begin{itemize}
          \item There is an event,
          \item We can count how many elements fall in that event
        \end{itemize}
    \item Probability
          \begin{itemize}
            \item a function that maps any event to a number between 0 and 1
            \item In all examples we have learned so far, we need \textbf{equally likely events} to calculate probability 
          \end{itemize}  
     \item Set operations, e.g., exclusion-Inclusion properties  
          \begin{itemize}
            \item For a complicated event, we can break it down into smaller pieces 
            \item Mostly addition and subtraction
          \end{itemize}
      \item \blue{Conditional probability}
      \begin{itemize}
            \item More complicated events
            \item More multiplication and division
     \end{itemize}         
  \end{itemize}
}

\frame{
  \frametitle{More review}
  \begin{itemize}
    \item Combinatorial is still important for most of the problems
    \item Counting and probability calculation are still going to be a key part
    \item Example: Toss two coins
          \begin{itemize}
            \item Counting with identical coins: 3 outcomes
            \item Counting with distinguishable coins: 4 outcomes
            \item Probability of seeing different outcomes: $1/2$
          \end{itemize}
     \item Why it does not matter if the coins are identical?
     \[
        P = P(\{HT\}) = P(\{1=H, 2=T\} \cup \{1=T, 2=H\})
     \]     
  \end{itemize}
}

\frame{
  \frametitle{More review}
  \begin{itemize}
    \item Why not $1/3$?
    \item In counting, we removed repeated outcomes depending on our sample space, i.e., $\{HT\}, \{TH\}$ are considered one outcome.
    \item In probability calculation, this leads to problems, as the outcomes we considered are not equally likely.
    \item So we can't use the ratio of the sizes of two sets to get probability.
    \item which means, \blue{when calculating probability, many times we need to put labels back in, so that we can define a sample space where each outcome is equally likely}. 
  \end{itemize}
}

%================================================================%
\section{Conditional probability}
\stepcounter{subsection}
\frame{
  \frametitle{Motivation}
  Consider the simple scenario of rolling 2 dice - we know there are 36 outcomes - What is the probability that the sum of the two numbers we get is 3? 
  \vspace{2cm}
  Will this change after we roll the first die?
  \vspace{3cm}
}

\frame{
  \frametitle{Conditional probability}
  Definition
  \vspace{5cm}
}

\frame{
\frametitle{Example}
}


\frame{
	\frametitle{Example: Russian roulette}
	Let’s play a game of Russian roulette. You are tied to your chair. Here’s a gun, a revolver. Here’s the barrel of the gun, six chambers, all empty. Now watch me as I put two bullets into the barrel, into two adjacent chambers. I close the barrel and spin it. I put a gun to your head and pull the trigger. Click. Lucky you! Now I’m going to pull the trigger one more time. Which would you prefer: that I spin the barrel first or that I just pull the trigger?
}

\frame{
	\frametitle{Multiplication rule}
}

\frame{
  \frametitle{Previous examples revisited}
}

\frame{
  \frametitle{Previous examples revisited}
}

\frame{
  \frametitle{Matching problem}
}


%================================================================%
\section{Bayes's Formula}
\stepcounter{subsection}

\frame{
  \frametitle{Bayes's formula}
  First Bayes's formula
  \vspace{6cm}
}
\frame{
  \frametitle{Bayes's formula}
  A special case

  \vspace{2cm}
  
    Second Bayes's formula
  \vspace{3cm}
}
\frame{
  \frametitle{Example}
  Flip a fair coin. If you toss Heads, roll 1 die. If you toss Tails, roll 2 dice. Compute the probability that you roll exactly one 6.
  \vspace{5cm}
}
\frame{
  \frametitle{Example}
  Roll a die, then select at random, without replacement, as many cards from the deck as the number shown on the die. What is the probability that you get at least one Ace?
  \vspace{4cm}
}

\frame{
  \frametitle{Coupon collector problem revisited}

}

\frame{
  \frametitle{Optimism in probability}
  A test of a disease presents a rate of 5\% false positives, and no false negatives. The disease strikes 1/1,000 of the population. People are tested at random, regardless of whether they are suspected of having the disease. A patient's test is positive. What is the probability of the patient being stricken with the disease?
  \vspace{3cm}

}


\frame{
  \frametitle{OJ Simpson}
  The following statement is from one of the lawyers of O.J., 
  \begin{quote}
  We knew we could prove, if we had to, that an infinitesimal percentage - certainly fewer than 1 of 2,500 - of men who slap or beat their domestic partners go on to murder them.
  \end{quote}
}

\frame{
  \frametitle{OJ Simpson}
}


\end{document}