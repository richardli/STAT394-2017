\documentclass[usenames,dvipsnames,12pt,compress, final, handout]{beamer}\usepackage[]{graphicx}\usepackage[]{color}
\usepackage{alltt} 
\let\Tiny=\tiny
\usepackage{eqnarray,amsmath}
% \usepackage{wasysym}
\usepackage{mathtools}
\usepackage{multirow}
\usepackage{booktabs}
\usepackage{tabularx}
\usepackage{array}
\usepackage{multirow}
% \usepackage{bigstrut}
\usepackage{graphicx}
\usepackage[round]{natbib}
\usepackage{bm}
\usepackage{tikzsymbols}
\setbeamersize{text margin left=15pt,text margin right=10pt}
\usepackage{tikzsymbols}
\usepackage{textcomp}
\usepackage{parskip}
\setbeamertemplate{navigation symbols}{}    
\setbeamertemplate{footline}[frame number]{}
\usepackage{soul}
\usetheme{Singapore}
\usecolortheme{lily}
%Misc Commands
\newcommand{\mbf}{\mathbf}
\newcommand{\lexp}{$\overset{\mbox{\tiny 0}}{e}$}
\newenvironment{wideitemize}{\itemize\addtolength{\itemsep}{5pt}}{\enditemize}


\newcommand{\bx}{{\bm x}}
\newcommand{\bX}{{\bm X}}
\newcommand{\by}{{\bm y}}
\newcommand{\bY}{{\bm Y}}
\newcommand{\bW}{{\bm W}}
\newcommand{\bG}{{\bm G}}
\newcommand{\bR}{{\bm R}}
\newcommand{\bZ}{{\bm Z}}
\newcommand{\bV}{{\bm V}}
\newcommand{\bL}{{\bm L}}
\newcommand{\bz}{{\bm z}}
\newcommand{\be}{{\bm e}}
\newcommand{\bgamma}{{\bm \gamma}}
\newcommand{\bbeta}{{\bm \beta}}
\newcommand{\balpha}{{\bm \alpha}}
\newcommand{\bSigma}{{\bm \Sigma}}
\newcommand{\bmu}{{\bm \mu}}
\newcommand{\btheta}{{\bm \theta}}
\newcommand{\bepsilon}{{\bm \epsilon}}
\newcommand{\bone}{{\bm 1}}
\newcommand{\bzero}{{\bm 0}}
\newcommand{\bC}{{\bm C}}
\newcommand{\bI}{{\bm I}}
\newcommand{\bA}{{\bm A}}
\newcommand{\bB}{{\bm B}}
\newcommand{\bQ}{{\bm Q}}
\newcommand{\bS}{{\bm S}}
\newcommand{\bD}{{\bm D}}
\newcommand{\cQ}{\mathcal{Q}}
\newcommand{\cR}{\mathcal{R}}
\newcommand{\cU}{\mathcal{U}}
\newcommand{\cI}{\mathcal{I}}
\newcommand{\cL}{\mathcal{L}}
\newcommand{\RR}{\mathbb{R}}
\newcommand{\orange}{\textcolor{Orange}}
\newcommand{\green}{\textcolor{green}}
\newcommand{\blue}{\textcolor{blue}}
\newcommand{\red}{\textcolor{red}}
\newcommand{\purple}{\textcolor{purple}}
\newcommand{\gray}{\textcolor{gray}}
\newcommand{\ok}{\nonumber}

% Adjust vertical spacing in lists
\makeatletter
\def\@listi{\leftmargin\leftmargini
            \topsep     8\p@ \@plus2\p@ \@minus2.5\p@
            \parsep     0\p@
            \itemsep  5\p@ \@plus2\p@ \@minus3\p@}
\let\@listI\@listi
\def\@listii{\leftmargin\leftmarginii
              \topsep    6\p@ \@plus1\p@ \@minus2\p@
              \parsep    0\p@ \@plus\p@
              \itemsep  3\p@ \@plus2\p@ \@minus3\p@}
\def\@listiii{\leftmargin\leftmarginiii
              \topsep    3\p@ \@plus1\p@ \@minus2\p@
              \parsep    0\p@ \@plus\p@
              \itemsep  2\p@ \@plus2\p@ \@minus3\p@}
\makeatother
% Dealing with fraile envrionment of beamer with codes
\newenvironment{xframe}[2][]
  {\begin{frame}[fragile,environment=xframe,#1]
  \frametitle{#2}}
  {\end{frame}}



\title{Stat 394 Probability I}
\subtitle{Discussion problems}

\author[]{Richard Li}
\date{\today}
\IfFileExists{upquote.sty}{\usepackage{upquote}}{}
\begin{document}
% \renewcommand{\itemize}[1][<+(1)->]{\olditemize[#1]}

\maketitle

%================================================================%
\frame{
  \frametitle{Problem 2}

  In 1990 - 1993, the National Basketball Association (NBA) draft lottery involves the 11 teams that had the worst won–lost records during the year. A total of 66 balls are placed in an urn. Each of these balls is inscribed with the name of a team: Eleven have the name of the team with the worst record, 10 have the name of the team with the second-worst record, 9 have the name of the team with the third-worst record, and so on (with 1 ball having the name of the team with the 11th-worst record). 
}
\frame{
  \frametitle{Problem 2}

A ball is then chosen at random, and the team whose name is on the ball is given the first pick in the draft of players about to enter the league. Another ball is then chosen, and if it ``belongs'' to a team different from the one that received the first draft pick, then the team to which it belongs receives the second draft pick. (If the ball belongs to the team receiving the first pick, then it is discarded and another one is chosen; this continues until the ball of another team is chosen.) Finally, another ball is chosen, and the team named on the ball (provided that it is different from the previous two teams) receives the third draft pick. The remaining draft picks 4 through 11 are then awarded to the 8 teams that did not “win the lottery,” in inverse order of their won–lost records. 
}

\frame{
  \frametitle{Problem 2}
For instance, if the team with the worst record did not receive any of the 3 lottery picks, then that team would receive the fourth draft pick. Let $X$ denote the draft pick of the team with the worst record. Find the probability mass function of $X$.
}

\frame{
  \frametitle{Problem 2}

}

\frame{
  \frametitle{Problem 5}
  You and your opponent both roll a fair die. If you both roll the same number, the game is repeated, otherwise whoever rolls the larger number wins. Let $X$ be the number of times the two dice have to be rolled before the game is decided. (a) Determine the probability mass function of $X$. (b) Compute the probability of you winning the game. (d) Assume that you get paid \$10 for winning in the first round, \$1 for winning in any other round, and nothing otherwise. Compute your expected winnings.

}

\frame{
  \frametitle{Problem 5}

}

\end{document}