\documentclass[usenames,dvipsnames,12pt,compress, final]{beamer}\usepackage[]{graphicx}\usepackage[]{color}
\usepackage{alltt} 
\let\Tiny=\tiny
\usepackage{eqnarray,amsmath}
% \usepackage{wasysym}
\usepackage{mathtools}
\usepackage{multirow}
\usepackage{booktabs}
\usepackage{tabularx}
\usepackage{array}
\usepackage{multirow}
% \usepackage{bigstrut}
\usepackage{graphicx}
\usepackage[round]{natbib}
\usepackage{bm}
\usepackage{tikzsymbols}
\setbeamersize{text margin left=15pt,text margin right=10pt}
\usepackage{tikzsymbols}
\usepackage{textcomp}
\usepackage{parskip}
\setbeamertemplate{navigation symbols}{}    
\setbeamertemplate{footline}[frame number]{}
\usepackage{soul}
\usetheme{Singapore}
\usecolortheme{lily}
%Misc Commands
\newcommand{\mbf}{\mathbf}
\newcommand{\lexp}{$\overset{\mbox{\tiny 0}}{e}$}
\newenvironment{wideitemize}{\itemize\addtolength{\itemsep}{5pt}}{\enditemize}


\newcommand{\bx}{{\bm x}}
\newcommand{\bX}{{\bm X}}
\newcommand{\by}{{\bm y}}
\newcommand{\bY}{{\bm Y}}
\newcommand{\bW}{{\bm W}}
\newcommand{\bG}{{\bm G}}
\newcommand{\bR}{{\bm R}}
\newcommand{\bZ}{{\bm Z}}
\newcommand{\bV}{{\bm V}}
\newcommand{\bL}{{\bm L}}
\newcommand{\bz}{{\bm z}}
\newcommand{\be}{{\bm e}}
\newcommand{\bgamma}{{\bm \gamma}}
\newcommand{\bbeta}{{\bm \beta}}
\newcommand{\balpha}{{\bm \alpha}}
\newcommand{\bSigma}{{\bm \Sigma}}
\newcommand{\bmu}{{\bm \mu}}
\newcommand{\btheta}{{\bm \theta}}
\newcommand{\bepsilon}{{\bm \epsilon}}
\newcommand{\bone}{{\bm 1}}
\newcommand{\bzero}{{\bm 0}}
\newcommand{\bC}{{\bm C}}
\newcommand{\bI}{{\bm I}}
\newcommand{\bA}{{\bm A}}
\newcommand{\bB}{{\bm B}}
\newcommand{\bQ}{{\bm Q}}
\newcommand{\bS}{{\bm S}}
\newcommand{\bD}{{\bm D}}
\newcommand{\cQ}{\mathcal{Q}}
\newcommand{\cU}{\mathcal{U}}
\newcommand{\cI}{\mathcal{I}}
\newcommand{\cL}{\mathcal{L}}
\newcommand{\orange}{\textcolor{Orange}}
\newcommand{\green}{\textcolor{green}}
\newcommand{\blue}{\textcolor{blue}}
\newcommand{\red}{\textcolor{red}}
\newcommand{\purple}{\textcolor{purple}}
\newcommand{\gray}{\textcolor{gray}}
\newcommand{\ok}{\nonumber}

% Adjust vertical spacing in lists
\makeatletter
\def\@listi{\leftmargin\leftmargini
            \topsep 		8\p@ \@plus2\p@ \@minus2.5\p@
            \parsep 		0\p@
            \itemsep	5\p@ \@plus2\p@ \@minus3\p@}
\let\@listI\@listi
\def\@listii{\leftmargin\leftmarginii
              \topsep    6\p@ \@plus1\p@ \@minus2\p@
              \parsep    0\p@ \@plus\p@
              \itemsep  3\p@ \@plus2\p@ \@minus3\p@}
\def\@listiii{\leftmargin\leftmarginiii
              \topsep    3\p@ \@plus1\p@ \@minus2\p@
              \parsep    0\p@ \@plus\p@
              \itemsep  2\p@ \@plus2\p@ \@minus3\p@}
\makeatother
% Dealing with fraile envrionment of beamer with codes
\newenvironment{xframe}[2][]
  {\begin{frame}[fragile,environment=xframe,#1]
  \frametitle{#2}}
  {\end{frame}}



\title{Stat 394 Probability I}
\subtitle{Lecture 1}

\author[]{Richard Li}
\date{\today}
\IfFileExists{upquote.sty}{\usepackage{upquote}}{}
\begin{document}
% \renewcommand{\itemize}[1][<+(1)->]{\olditemize[#1]}

\maketitle

%================================================================%
\section{Count and Permutation}
\stepcounter{subsection}
\frame{
	\frametitle{Counting}
	A silly example: 
	\begin{itemize}
		\item There are $4$ weeks in the summer quarter.
		\item Each week there are $3$ lectures of Stat 394.
		\item How many lectures there are in total?
	\end{itemize}
	% \pause
	% A less silly but still simple example:
	% \begin{itemize}
	% 	\item 
	% \end{itemize}
}
\frame{
	\frametitle{Principles of counting}
	\newtheorem{counting}{The basic principle of counting}
	\begin{counting}
		\vspace{2cm}  
	\end{counting} 
	\newtheorem{counting2}{The generalized basic principle of counting}
	\begin{counting2}
		\vspace{2cm}    
	\end{counting2} 
}

\frame{
\frametitle{Example: counting}
\begin{itemize}
	\item How many different 6-digit passwords are possible if the first 2 places are letters, and the final 4 are numbers?
	\pause
	\begin{itemize}
		\item $26 \times 26 \times 10\times 10\times 10\times 10=6,760,000$
	\end{itemize}
	\pause
	\item What if each number of letter can only be used once?
	\pause
	\begin{itemize}
		\item $26 \times 25 \times 10\times 9\times 8\times 10=468,000$
	\end{itemize}
\end{itemize}
}

\frame{
	\frametitle{Permutation}
	\begin{itemize}
		\item In a basketball game, $5$ players from the starting lineup are introduced before the game.
		\item How many possible ordered arrangements could there be?
	\end{itemize}
}
\frame{
	\frametitle{Permutation}
	In general,
	\vspace{3cm}
}
\frame{
	\frametitle{Permutation}
	\begin{itemize}
		\item Back to the basketball example, since both teams are introduced, how many different ordered arrangements of the $10$ players  could there be?
		\pause
		\begin{itemize}
		 	\item $5! \times 5! = 14,400$ 
		 \end{itemize} 
		 \pause
		 \item Among the five players, assume there are two guards, two forwards, and one center, how many different ordered arrangements could there be in terms of the positions?
		 \pause
		\begin{itemize}
		 	\item $\frac{5!}{2!2!} = 30$ 
		 \end{itemize} 
		 \pause
		 \item What if there are two guards, and three forwards?
		 \pause
		\begin{itemize}
		 	\item $\frac{5!}{2!3!} = 10$ 
		 \end{itemize} 
		 \pause
	\end{itemize}
}
\frame{
	\frametitle{Permutation}
	In general, for $n$ object, of which $n_1$ are alike, $n_2$ are alike, ..., $n_r$ are alike, the total number of different permutations is 
	\[
		\frac{n!}{n_1!n_2! \cdots n_r!}
	\]
}

%================================================================%
\section{Combinations}
\stepcounter{subsection}
\frame{
	\frametitle{Combinations}
	\begin{itemize}
		\item Now, consider again counting the permutations of $GGFFF$ 
		\item An alternative view: \pause
		\begin{itemize}
			\item find $2$ out of $5$ slots to put $G$.
			\item ${5\choose 2} = \frac{5!}{3! \times 2!}$
		\end{itemize}
		\item In general,
		\vspace{2cm}
	\end{itemize}
}

\frame{
	\frametitle{Combinations}
	Some special cases or convention:
	\begin{itemize}
		\item ${n\choose 0} = 1$
		\item ${n\choose n} = 1$
		\item ${n\choose k} = 0$, if $k < 0$ or $k > n$
	\end{itemize}
}

\frame{
\frametitle{Example}
	You are at a Poke place, from $5$ kinds of fish and $10$ choices of toppings, how many different combinations consisting of $3$ fish and $5$ toppings can be formed (assume you can't choose the same fish more than once)
	\vspace{3cm}

}
\frame{
\frametitle{Example}
	What if $2$ of the toppings are very spicy, and you don't want to have them both together?
	\vspace{4cm}

}
\frame{
	\frametitle{Multinomial coefficients}
	A set of $n$ distinct items is to be divided into $r$ distinct groups of size $n_1, n_2, ..., n_r$, where $\sum_{i=1}^r n_i = n$. How many different divisions are possible?
	\vspace{3cm}
}

\frame{
	\frametitle{Example}
	In a knockout tournament involving $64$ players. In each round, the players are divided into pairs, and the winners go on the next round.  How many possible outcomes are there for the first round? 
	\vspace{4cm}
}
\frame{
	\frametitle{Example}
	
}


\frame{
	\frametitle{Example}
	
}
\frame{
	\frametitle{Summary}
	\begin{itemize}
		
		\item Permutation and combination are strongly related, as we have seen in examples.
		\item We will see more of these in the chapters to come.
		\item Read Chapter 1 on your own (especially multinomial theorem in Sec 1.5).
		\item Homework 1 due next Monday.  
	\end{itemize}
}



\end{document}